\documentclass[12pt]{bsc}

\usepackage[T1,MeX]{polski}
\usepackage[utf8]{inputenc}
\usepackage[T1]{fontenc}
\usepackage{txfonts}

\usepackage{graphicx}
\usepackage{fancyvrb}
\usepackage{pdfpages}

\usepackage{enumitem}
\usepackage{listings}
\usepackage{caption}

\usepackage{alltt}

%sposob na zablokowanie obrazka na stronie
%\FloatBarrier (wcześniej dołącz \usepackage{placeins})
\usepackage{placeins}


%caption listingów
\DeclareCaptionFont{black}{ \color{black} }
\DeclareCaptionFormat{listing}{
  {
    \parbox{\textwidth}{\hspace{12pt}#1#2#3}
  }
}
\captionsetup[lstlisting]{ format=listing, labelfont=black, textfont=black, singlelinecheck=false, margin=0pt, font={bf,footnotesize} }

%definicja custom listingów
\lstdefinestyle{custom}{
  breaklines=true,
  xleftmargin=\parindent,
  showstringspaces=false,
  basicstyle=\fontsize{10}{11}\selectfont,
}

%pozwala na używanie znaku dolara($) w listingach
\lstset{
    mathescape=false
}

\usepackage[pdftex,pdfstartview=FitH,unicode]{hyperref}
% "klikalny" spis treści
\hypersetup{
    colorlinks,
    citecolor=black,
    filecolor=black,
    linkcolor=black,
    urlcolor=black
}

%https://tex.stackexchange.com/questions/36880/insert-a-blank-page-after-current-page
\usepackage{afterpage}

\newcommand\blankpage{%
    \null
    \thispagestyle{empty}%
    \addtocounter{page}{-1}%
    \newpage}
    


\graphicspath{{./img/}}

\selecthyphenation{polish}


% ********* documents meta data **********
\makeatletter

\def\@title{Analiza porównawcza nowoczesnych rozwiązań technologicznych aplikacji SPA za pomocą jednakowej aplikacji zaimplementowanej w każdym z wybranych rozwiązań}
\def\@engTitle{Comparative analysis of modern technological solutions of SPA applications using the same application implemented in each of the selected solutions}
\def\@author{Daniel Słaby}
\def\@promoter{Dr inż. Grzegorz Debita}
\def\@when{2020-05-08}
\def\@year{2020}
\def\@album{6781}

\makeatother

\renewcommand{\partname}{}

\begin{document}
  \include{front}
  \tableofcontents
  
  \chapter{Wstęp}

\section{Wprowadzenie}

Technologia front-end ewoluuje w zastraszającym tempie. 20 października 2010 rozpoczęła się nowa era technologii webowych wraz z wydaniem biblioteki AngularJS. Był to swoisty początek aplikacji SPA ( Single Page Application ) i całkowitej zmiany podejścia do tworzenia webowych aplikacji klienckich. 

Dzisiaj większość narzędzi jakie używamy w życiu codziennym albo w pełni znajduje się w przeglądarce albo ma już na niej swoje odpowiedniki. Znaczna część aplikacji z których korzystamy na co dzień wykorzystuje te narzędzia jako rdzeń swojego działania. Ze znanych aplikacji z jakich korzysta na co dzień użytkownik możemy wymienić jako przykład odpowiednio:
React: Facebook, Instagram, Netflix, WhatsApp
Angular 2: GitHub Community Forum, Microsoft Office Home, Google Marketing Platform
VueJS: FontAwesome, ChatWoot, Moonitor, Leave Dates


\section{Cel pracy}

W pracy tej, przeanalizowano trzy najpopularniejsze rozwiązania do tworzenia aplikacji SPA. Każde z nich pozwala na stworzenie takiej samej aplikacji z punktu widzenia końcowego użytkownika. Każda z nich jednak w inny sposób rozwiązuje problemy zarządzania zasobami oraz optymalizacji renderowania. W tym celu skonstruuje narzędzie pozwalające na badanie dynamicznych aplikacji webowych i postaram się odpowiedzieć na pytanie, czy różnica wydajności pomiędzy frameworkiami jest tak duża, że powinna mieć wpływ na wybór konkretnego rozwiązania w ogólnym przypadku?

\section{Motywacja}

Za motywacją do stworzenia tejże pracy stały dwa główne powody. Pierwszym z nich jest chęć porównania trzech najczęściej wybieranych narzędzi w środowisku front-end. Każde z nich rozwiązuje ten sam problem, jednak nie jest do jasne, jak wygląda ich wydajność w przypadku użytkowania aplikacji. Jest to bowiem problem który trudno zbadać, co prowadzi nas do drugiego powodu. Badanie aplikacji typu SPA w momencie ich działania jest niezwykle trudnym zadaniem, dlatego też, celem praktycznym pracy jest stworzenie narzędzia do rozwiązania tego problemu.


\section{Zakres}

Zadanie będzie polegać na zaprojektowaniu odpowiedniej aplikacji która zawiera kilka znanych problemów wydajnościowych występujących w świecie aplikacji SPA. Następnie implementacja takiej aplikacji w każdym z wybranych rozwiązań. Kolejno na specjalnie stworzonej maszynie wirtualnej z ograniczonymi zasobami, wykonane zostaną testy wydajnościowe za pomocą wbudowanego profilera. 

Mając już konkretne dane, wyniki zostaną przeanalizowane porównując które rozwiązanie jest najwydajniejsze w konkretnym przypadku co pozwoli na odpowiedź na pytanie: czy różnica wydajności pomiędzy narzędziami jest tak duża, że powinna mieć wpływ na wybór konkretnego rozwiązania w ogólnym przypadku?

  
  \part{Część przeglądowa}  
  \chapter{Część przeglądowa}

\begin{itemize}
	\item JSON \cite{JSON} (JavaScript Object Notation) - jest to tekstowy format wymiany danych, bazującym na podzbiorze języka JavaScript. JSON przechowuje dane określone przez standard ECMA-262 3-cia edycja - Grudzień 1999.
	\item Transpilacja \cite{Transpilator} - Transpilatory lub kompilatory typu źródło-źródło to narzędzia, które odczytują kod źródłowy napisany w jednym języku programowania i wytwarzają równoważny kod w innym języku. Języki, które piszesz, które są transpilowane na JavaScript, są często nazywane językami kompilacji do JS i mówi się, że są ukierunkowane na JavaScript.
	\item SPA \cite{SPA} (także Single Page Application) - jednostronna aplikacja internetowa, czyli taka, która posiada tylko jeden plik HTML. Taka aplikacja nie przeładowuje strony w trakcie użytkowania. Może w tym celu korzystać z technologii AJAX lub innych dostępnych w przeglądarkach internetowych. Logika aplikacji SPA napisana jest w JavaScript lub w języku transpilowanym do języka JavaScript Framework.
\end{itemize}


\section{Słowniczek Pojęć}

\section{Przegląd prac o podobnej tematyce}

W części tej zajmiemy się przeglądem dostępnych rozwiązań oraz prac dotyczących pomiaru wydajności aplikacji internetowych z naciskiem na aplikacje typu SPA.

\subsection{Comparison of Single-Page Application Frameworks}

Pierwszą pozycją jest praca autorstwa Pan Eric'a Molina z instytutu KTH Royal Institute of Technology w Sztokholmie \cite{Molin}.
Autor stara się porównać frameworki na pełnej płazczyźnie ich złożoności.
W jego pracy znaleźć możemy pełen spis kryteriów oceny frameworków wraz z wagami przypisanymi do cech określonymi za pomocą wywiadu środowiskowego jaki autor przeprowadził na grupie 10 profesjonalnych deweloperów mających doświadczenie w tych zagadnieniach przedstawionych na rysunku \ref{fig:rysunek_1}.

\begin{figure}[!ht]
    \centering
    \includegraphics[width=12cm]{rysunek_1.png}
    \caption{Grafika przedstawiająca ocenę wpływu wybranego elementu na atrakcyjność danego narzędzia \cite{Molin}}
    \label{fig:rysunek_1}
\end{figure}




W pracy tej, porównywanymi frameworkami są AngularJS, Angular2 oraz React.
I tutaj wyróżnić możemy już pierwsze różnice pomiędzy pracami. Praca Pana Molina została wykonana w 2016 roku.
JavaScript jest jednym z najszybciej rozwijających  się oraz najczęściej spotykanych technologii webowych na świecie \cite{octoverse}.
Przez okres 4 lat, firmy masowo porzuciły technologie AngularJS na rzecz Angular 2 oraz Reacta.
Dodatkowo, w między czasie do wyścigu dołączył Vue.Js \cite{vue}.
Przez ten okres, także implementacja tychże rozwiązań znacznie dojrzała.
Angular z wersji 2 ewoluował już do wersji 9 \cite{angular-changelog}.
Także React ewoluował z wersji 0.14.8 (użyta w pracy  Pana Molina) do wersji 16.13.1 \cite{react-changelog}.
Najważniejszą zmianą którą należy tutaj wspomnieć jest zmiana silnika reacta nosząca nazwę React Fiber która nastąpiła w wersji 16.0.0.
Był to kamień milowy, i z punktu widzenia badania wydajności, jest to całkowicie nowe rozwiązanie. 

Następną sekcją która jest dla nas interesująca jest sekcja 5.2 w której autor prezentuje wyniki swoich testów.
Wartym zauważenia jest wersja przeglądarki chrome, która wynosi 50.0.26661.102m.
Najnowsza wersja chrome, w czasie pisania tego tekstu, to wersja \textit{80.0.3987.16} czyli aż o 30 wersji nowszej. 

W pierwszym teście, autor porównuje wydajność pomiędzy rozwiązaniami na przykładzie ładowania elementów od 10 aż do 5000 przedstawionym na rysunku \ref{fig:rysunek_2}.

\begin{figure}[!ht]
    \centering
    \includegraphics[width=12cm]{rysunek_2.png}
    \caption{Grafika przedstawiająca wzrost czasu ładowania elementów od 10 do 5000 przy użyciu AngularJS Angular2 oraz React \cite{Molin}}
    \label{fig:rysunek_2}
\end{figure}

Z badania tego, możemy zauważyć jak bardzo AngularJS różni się od pozostałych, nowszych rozwiązań. Do pierwszych 100 elementów, wyniki są zbliżone, natomiast powyżej tej wartości, mamy nagłą zmianę stopnia nachylenia funkcji. Przyczyną takiego zachowania jest fakt, iż AngularJS nigdy nie był zoptymalizowany dla takich przypadków użycia podczas, gdy React oraz Angular2 już tak.
W drugim teście, autor bada czas potrzebny do załadowania określonej ilości elementów. Wyniki przedstawiono na rysunku \ref{fig:rysunek_3}.

\begin{figure}[!ht]
    \centering
    \includegraphics[width=12cm]{rysunek_3.png}
    \caption{Grafika przedstawiająca wykres zależności czasu do ilości edytowanych elementów na stronie \cite{Molin}}
    \label{fig:rysunek_3}
\end{figure}

Widzimy znaczną dysproporcję pomiędzy każdym z rozwiązań. Z jednej strony AngularJS przypomina funkcję wykładniczą gdzie po przeciwnej stronie Angular2 zachowuje się jak funkcja liniowa. Pokazuje to jak bardzo istotne zmiany zaszły pomiędzy zmianą generacji oraz wpływ doświadczenia jakie zebraliśmy w okresie panowania frameworka AngularJS. Wszak AngularJS był pierwszym podejściem do generycznego narzędzia dla dynamicznych aplikacji internetowych.
W ostatnim teście wydajnościowym, autor porównuje czas ładowania się aplikacji. Wyniki przedstawiono na rysunku \ref{fig:rysunek_4}.

\begin{figure}[!ht]
    \centering
    \includegraphics[width=12cm]{rysunek_4.png}
    \caption{Grafika przedstawiająca wykres czasu ładowania aplikacji \cite{Molin}}
    \label{fig:rysunek_4}
\end{figure}

O ile autor faktycznie porównuje wydajność rozwiązań, nie określa on w sposób klarowny jak zaimplementowano mechanizm dodawania oraz aktualizacji elementów w pierwszym oraz drugim teście.
Są to informacje bardzo istotne, gdyż każdy z frameworków działa w inny sposób, i ma to bezpośredni wpływ na otrzymane wyniki.
Dodatkowo, autor nie podaje błędu pomiaru \cite{probalistyka} który wyznaczył podczas przeprowadzania badania.
Jest to bardzo istotna informacja, gdyż frameworki takie opierają się na całej piramidzie warstw oprogramowania które współdzielą zasoby komputera z innymi procesami.
Ponadto auto nie zastosował żadnej warstwy abstrakcji nad systemem operacyjnym co powoduje, że wyniki mogą znacząco różnić się pomiędzy różnymi wersjami systemu.

\subsection{Lighthouse}



\subsection{Porównanie narzędzi do tworzenia aplikacji typu SPA na przykładzie Angular2 i React}


  \part{Część praktyczna}
  \chapter{Część praktyczna}


\begin{figure}[!ht]
    \centering
    \includegraphics[width=12cm]{rysunek_7.png}
    \caption{Ilustracja przedstawiająca cykl życia statycznej strony internetowej}
    \label{fig:rysunek_7}
\end{figure}

\begin{figure}[!ht]
    \centering
    \includegraphics[width=12cm]{rysunek_8.png}
    \caption{Ilustracja przedstawiająca cykl działania aplikacji SPA}
    \label{fig:rysunek_8}
\end{figure}

\begin{figure}[!ht]
    \centering
    \includegraphics[width=12cm]{rysunek_9.png}
    \caption{Grafika przedstawiająca mechanizm działania aplikacji dynamicznych}
    \label{fig:rysunek_9}
\end{figure}

\begin{figure}[!ht]
    \centering
    \includegraphics[width=12cm]{rysunek_10.png}
    \caption{Grafika przedstawiająca kolejność faz renderowania w przeglądarce}
    \label{fig:rysunek_10}
\end{figure}


%czyści puste strony
\let\cleardoublepage\clearpage
  
  \chapter{Podsumowanie}

Celem pracy było stworzenie narzędzia pozwalającego na pomiar akcji aplikacji zachodzących już po załadowaniu strony oraz pomiar różnych interakcji ze stroną na przykładzie trzech wiodących rozwiązań na rynku.
Stworzenie narzędzia było niemałym wyzwaniem, gdyż problem pomiaru dynamicznych aplikacji nie jest problemem trywialnym i wymaga szeregu narzędzi współpracujących ze sobą.
Powoduje to także, iż sam pomiar będzie zanieczyszczony przez niedokładności narzędzie Selenium.
Jesteśmy w stanie jednak przeciwdziałać niektórym czynnikom mającym negatywny wpływ na pomiar, dzięki zastosowaniu warstwy izolacja systemu badającego od faktycznego systemu operacyjnego hosta badania dzięki użyciu platformy Docker.
Także poprawne przygotowanie badania na przykład poprzez przeładowanie strony i wyczyszczenie pamięci podręcznej przeglądarki poprawia stabilność i dokładność testów.
Istotnym podczas testów jest wielokrotny pomiar wartości tak, abyśmy mogli wyliczyć graniczne wartości pomiarów. Dzięki temu i wiedzy uzyskanej podczas studiów możemy przeprowadzić poprawnę analizę wyników.

Same badania wykazały, iż Angular2 najczęściej jest najszybszym narzędziem, jednak bardzo często odchylenie standardowe wskazywało, iż także istnieje duży rozrzut wartości wokół średniej.
React jest najbardziej stabilnym rozwiązaniem, gdyż w dwóch badaniach uzyskał odchylenie standardowe na poziomie poniżej 2 milisekund. W reszcie przypadków, różnica pomiędzy Vue i Reactem mieściła się w granicy błędu pomiarowego.

Podsumowując -najważniejszym wnioskiem który udało się potwierdzić dzięki przeprowadzonym badaniom jest wykazanie, iż różnica pomiędzy przedstawionymi rozwiązaniami nie przekracza 20 milisekund dla pojedynczego zadania.
Jest to istotny argument podczas dyskusji nad przewagą konkretnych narzędzi pomiędzy sobą.
Jest to wniosek zgodny także z pracami cytowanych autorów, iż najistotniejsze jest środowisko programistyczne dostarczone wraz z narzędziem oraz popularność narzędzia która bezpośrednio ma wpływ na przykład
na ilość informacji dostępnych w internecie na temat częstych problemów niżeli wydajność pojedynczego narzędzia w skali mikro.


%czyści puste strony
\let\cleardoublepage\clearpage

  \pagestyle{plain}
  \cleardoublepage
\phantomsection
\addcontentsline{toc}{chapter}{Bibliografia}


\begin{thebibliography}{99}

  \bibitem{JSON}
   JSON.org, https://www.json.org/
  
  \bibitem {Transpilator}
  Peleke Sengstacke
  \emph{JavaScript Transpilers: What They Are and Why We Need Them},
  Kwiecień 25, 2016

  \url{https://scotch.io/tutorials/javascript-transpilers-what-they-are-why-we-need-them}

\bibitem {SPA}
Sławomir Kołodziej
\emph{What Are Single Page Applications? What Is Their Impact on Users’ Experience and Development Process?},
3 Lipiec 2019

\url{https://www.netguru.com/blog/what-are-single-page-applications}

\bibitem {Molin}
Eric Molin, 
\emph{Comparison of Single-Page Application Frameworks},
Instytut KTH w  Sztokholmie

\url{https://pdfs.semanticscholar.org/fa9f/f75f32de61cddafa8805ea433d4d8a0e20da.pdf}

 \bibitem {octoverse}
https://octoverse.github.com/

 \bibitem {vue}
Maja Nowak, Reasons,
\emph{Why Vue.js Is Getting More Traction Every Month},
19 Grudzień 2018, 

\url{https://www.monterail.com/blog/reasons-why-vuejs-is-popular}

\bibitem {angular-changelog}

\url{https://github.com/angular/angular/blob/master/CHANGELOG.md}

 \bibitem {react-changelog}
 
\url{https://github.com/facebook/react/blob/master/CHANGELOG.md}

 \bibitem {probalistyka}
\emph{Probabilistyczny opis błędu jako podstawa definiowania niepewności pojedynczego wyniku pomiaru}

\url{http://yadda.icm.edu.pl/baztech/element/bwmeta1.element.baztech-article-BSW4-0034-0011}

 \bibitem {lighthouse}
Google Developers,
\emph{Lighthouse}

\url{https://developers.google.com/web/tools/lighthouse}

\bibitem {you-dont-know-js}
Kyle Simpson
\emph{You Don't Know JS: Async \& Performance}

\url{http://cdn.lxqnsys.com/05_You_Don't\%20_Know_JS_Async_\&_Performance.pdf}

 \bibitem {polibuda}
Jadwiga Kalinowska, Beata Pańczyk,
\emph{Porównanie narzędzi do tworzenia aplikacji typu SPA na przykładzie Angular2 i React},
Politechnika Lubelska, Instytut Informatyki

\url{http://yadda.icm.edu.pl/yadda/element/bwmeta1.element.baztech-5c6271b4-27e3-42d0-9762-a240dc3a9973}

\bibitem {rendering-performance}
Paul Lewis,
\emph{Google Developers, Rendering Performance},

\url{https://developers.google.com/web/fundamentals/performance/rendering}

 \bibitem {react-lists}
 \emph{ReactJS,  Lists and Keys},

\url{https://reactjs.org/docs/lists-and-keys.html}

 \bibitem {virtualdom}
\emph{ReactJS, Virtual DOM and Internals}, 

\url{https://reactjs.org/docs/faq-internals.html}

 \bibitem {reconcilation}
ReactJS, Reconciliation,

\url{https://reactjs.org/docs/reconciliation.html}

 \bibitem {gnu}
GNU.org, 
\emph{Makefile},

\url{https://www.gnu.org/software/make/manual/html_node/Introduction.html}

 \bibitem {npm}
 NpmJs.com,
\emph{npm | build amazing things},

\url{https://www.npmjs.com/}

 \bibitem {react-perf}
 ReactJS,
\emph{Optimizing Performance},

\url{https://reactjs.org/docs/optimizing-performance.html}

 \bibitem {kontenery}
magnifier.pl,
\emph{Konteneryzacja - czym jest i dlaczego staje się tak popularna?},
24 Październik. 2019,

\url{https://magnifier.pl/konteneryzacja-docker-kubernetes/}

 \bibitem {promise}
MDN web docs,
\emph{Promise - JavaScript},

\url{https://developer.mozilla.org/en/docs/Web/JavaScript/Reference/Global_Objects/Promise}

 \bibitem {web-server}
MDN web docs,
\emph{What is a web server?},
18 Stycznia 2019,

\url{https://developer.mozilla.org/en-US/docs/Learn/Common_questions/What_is_a_web_server}

 \bibitem {docker}
Docker Docs,
\emph{Docker},

\url{https://docs.docker.com/get-started/#docker-concepts}

 \bibitem {nginx-windows}
Nginx,
\emph{Nginx for Windows},

\url{http://nginx.org/en/docs/windows.html}

 \bibitem {nginx-linus}
Nginx, 
\emph{Linux packages},

\url{http://nginx.org/en/linux_packages.html}

 \bibitem {docker-alpine}
Wiki Alpine Linux, 
\emph{Docker},

\url{https://wiki.alpinelinux.org/wiki/Docker}

 \bibitem {ubuntu}
Docker Hub,
\emph{Ubuntu},

\url{https://hub.docker.com/_/ubuntu}

 \bibitem {rail-model}
Meggin Kearney, Addy Osmani, Kayce Basques, Jason Miller,
\emph{Measure Performance with the RAIL Model | Web Fundamentals},
12 Luty. 2019,

\url{https://developers.google.com/web/fundamentals/performance/rail}

 \bibitem {ruby}
Jesus Castello, 
\emph{Ruby Templating Engines: ERB, HAML \& Slim - RubyGuides},

\url{https://www.rubyguides.com/2018/11/ruby-erb-haml-slim/}

 \bibitem {threads}
Whatsabyte,
\emph{What Are Threads in a Processor?},
24 Sierpień 2018,

\url{https://whatsabyte.com/blog/processor-threads/}

 \bibitem {flaky-cypress}
Dimiter Petrov,
\emph{A tale of flaky Cypress tests},
24 Październik 2019,

\url{https://dimiterpetrov.com/blog/a-tale-of-flaky-cypress-tests/}

 \bibitem {selenium}
Selenium Dev,
\emph{The Selenium project and tools},

\url{https://www.selenium.dev/documentation/en/introduction/the_selenium_project_and_tools/}

 \bibitem {docker-compose}
Docker Docs,
\emph{Overview of Docker Compose},

\url{https://docs.docker.com/compose/.}

 \bibitem {docker-compose-network}
Docker Documentation,
\emph{Networking in Compose}.

\url{https://docs.docker.com/compose/networking/}

 \bibitem {mozilla-memory}
MDN - Mozilla,
\emph{Memory Management},
4 Marca 2020,

\url{https://developer.mozilla.org/en-US/docs/Web/JavaScript/Memory_Management}

 \bibitem {mozilla-perf}
MDN - Mozilla,
\emph{Performance - Web APIs},
30 Styczeń 2020,

\url{https://developer.mozilla.org/en-US/docs/Web/API/Performance}

\bibitem {mozilla-high-res-api}
MDN - Mozilla,
\emph{DOMHighResTimeStamp - Web APIs},
19 Lusty 2020,

\url{https://developer.mozilla.org/en-US/docs/Web/API/DOMHighResTimeStamp}

 \bibitem {mdn-first-paint}
MDN - Mozilla,
\emph{First paint - MDN Web Docs Glossary},
10 Marca. 2020,

\url{https://developer.mozilla.org/en-US/docs/Glossary/First_paint}


\end{thebibliography}


% Opis bibliograficzny wydawnictwa zwartego (książki) składa się z następujących pozycji [7]:
% autorzy (nazwisko + inicjały imion), tytuł (kursywa bez cudzysłowu), nazwa wydawnictwa,
% miejsce wydania, rok wydania (w nawiasach). Poszczególne części opisu powinny być
% oddzielone przecinkami. Przy dużej liczbie autorów można podać dane pierwszego autora z frazą
% „et al.” [5].
% 
% Opis artykułu w czasopiśmie [2,9]: autorzy (nazwisko + inicjały imion), tytuł (kursywa bez
% cudzysłowu), nazwa czasopisma, wolumin, numer, rok wydania (w nawiasach), strony „od–do”
% przedzielone znakiem półpauzy (Alt+0150), bez spacji w środku.
% 
% Opis referatu w materiałach konferencyjnych lub rozdziału pracy zbiorowej [4]: autorzy
% referatu (nazwisko + inicjały imion), tytuł referatu (kursywa bez cudzysłowu), słowo (w:),
% redaktorzy pracy zbiorowej (nazwisko + inicjały imion), słowo (red.), tytuł pracy zbiorowej lub
% dane konferencji (czcionka prosta), wydawnictwo, miejsce wydania, rok wydania (w nawiasach),
% strony „od–do” przedzielone znakiem półpauzy (Alt+0150), bez spacji w środku. Jeżeli
% opisywana praca jest częścią serii wydawniczej, należy dodać jej nazwę oraz numer woluminu
% pomiędzy tytułem pracy zbiorowej i nazwą wydawnictwa [6]. Możliwe jest także zastosowanie
% skróconego opisu referatu w materiałach konferencyjnych – bez podawania redaktorów i tytułu
% pracy zbiorowej [1].
% 
% W opisie materiałów publikowanych elektronicznie (np. specyfikacji, dokumentacji
% technicznej) należy umieścić dane autora lub nazwę producenta (jeśli nie ma podanego autora),
% tytuł dokumentu, ewentualnie opis rodzaju dokumentu (np. podręcznik użytkownika), wydawcę
% i rok wydania (w nawiasach) [8]. Opis strony internetowej składa się z danych autora albo nazwy
% organizacji publikującej, tytułu dokumentu albo nazwy serwisu, jego adresu URL (bez
% podkreślenia i niebieskiego koloru), roku opublikowania oraz daty odczytania dokumentu4 [3].
% 
% 1. Agrawal R., Srikant R., Fast Algorithms for Mining Association Rules, Proceedings
% of the Twentieth International Conference on Very Large Databases, Santiago, Chile
% (1994)
% 2. Bacchus F., Grove A., Halpern J., Koller D., From statistical knowledge bases
% to degrees of belief, Artificial Intelligence, Vol. 87, No. 1–2 (1996) 75–143
% 3. Brown D., A Beginners Guide to UML. Part 1–2., Dunstan Thomas Consulting,
% http://consulting.dthomas.co.uk (2002), (odczytano 10 września 2008 r.)
% 4. Deogun J., Jiang L., Comparative Evaluation on Concept Approximation Approaches,
% (w:) Kwaśnicka H., Paprzycki M. (red.), Proceedings of the Fifth International
% Conference on Intelligent Systems Design and Applications, IEEE Computer Society
% Press, Washington, Brussels, Tokyo (2005) 438–443
% 5. Fagin R. et al., Reasoning About Knowledge, MIT Press, Cambridge, USA (1995)
% 6. Kazakov D., Kudenko D., Machine Learning and Inductive Logic Programming for
% Multi-Agent Systems, (w:) Luck M., Marik V., Stepankova O. (red.), Multi-Agent
% Systems and Applications, Lecture Notes in Artificial Intelligence (LNAI), Vol. 2086,
% Springer-Verlag, Berlin Heidelberg (2001) 246–270
% 7. Kerninghan B.W., Ritchie D.M., Język ANSI–C, Wydawnictwa Naukowo-Techniczne,
% Warszawa (1994)
% 8. Microsoft, Books On-Line, dokumentacja elektroniczna systemu MS SQL Server 2000
% Enterprise Edition, Microsoft Corporation (2000)
% 9. Perry P., Walnum C., Pierwsze kroki Telefonii GSM, Kwartalnik Elektroniki
% i Telekomunikacji, Vol. 43, No. 3 (1997) 421–430

  
  \cleardoublepage
\phantomsection
\addcontentsline{toc}{chapter}{Spis rysunków}
\listoffigures

%czyści puste strony
\let\cleardoublepage\clearpage
   
  \include{Oswiadczenie_udostepnienie}
  
  \include{Oswiadczenie_autorskie}
\end{document}
