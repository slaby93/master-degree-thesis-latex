\cleardoublepage
\phantomsection
\addcontentsline{toc}{chapter}{Bibliografia}


\begin{thebibliography}{99}

  \bibitem{JSON}
   JSON.org, https://www.json.org/
  
  \bibitem {Transpilator}
  Peleke Sengstacke
  \emph{JavaScript Transpilers: What They Are and Why We Need Them},
  Kwiecień 25, 2016

  \url{https://scotch.io/tutorials/javascript-transpilers-what-they-are-why-we-need-them}

\bibitem {SPA}
Sławomir Kołodziej
\emph{What Are Single Page Applications? What Is Their Impact on Users’ Experience and Development Process?},
3 Lipiec 2019

\url{https://www.netguru.com/blog/what-are-single-page-applications}

\bibitem {Molin}
Eric Molin, 
\emph{Comparison of Single-Page Application Frameworks},
Instytut KTH w  Sztokholmie

\url{https://pdfs.semanticscholar.org/fa9f/f75f32de61cddafa8805ea433d4d8a0e20da.pdf}

 \bibitem {octoverse}
https://octoverse.github.com/

 \bibitem {vue}
Maja Nowak, Reasons,
\emph{Why Vue.js Is Getting More Traction Every Month},
19 Grudzień 2018, 

\url{https://www.monterail.com/blog/reasons-why-vuejs-is-popular}

\bibitem {angular-changelog}

\url{https://github.com/angular/angular/blob/master/CHANGELOG.md}

 \bibitem {react-changelog}
 
\url{https://github.com/facebook/react/blob/master/CHANGELOG.md}

 \bibitem {probalistyka}
\emph{Probabilistyczny opis błędu jako podstawa definiowania niepewności pojedynczego wyniku pomiaru}

\url{http://yadda.icm.edu.pl/baztech/element/bwmeta1.element.baztech-article-BSW4-0034-0011}

 \bibitem {lighthouse}
Google Developers,
\emph{Lighthouse}

\url{https://developers.google.com/web/tools/lighthouse}

\bibitem {you-dont-know-js}
Kyle Simpson
\emph{You Don't Know JS: Async \& Performance}

\url{http://cdn.lxqnsys.com/05_You_Don't\%20_Know_JS_Async_\&_Performance.pdf}

 \bibitem {polibuda}
Jadwiga Kalinowska, Beata Pańczyk,
\emph{Porównanie narzędzi do tworzenia aplikacji typu SPA na przykładzie Angular2 i React},
Politechnika Lubelska, Instytut Informatyki

\url{http://yadda.icm.edu.pl/yadda/element/bwmeta1.element.baztech-5c6271b4-27e3-42d0-9762-a240dc3a9973}

\bibitem {rendering-performance}
Paul Lewis,
\emph{Google Developers, Rendering Performance},

\url{https://developers.google.com/web/fundamentals/performance/rendering}

 \bibitem {react-lists}
 \emph{ReactJS,  Lists and Keys},

\url{https://reactjs.org/docs/lists-and-keys.html}

 \bibitem {virtualdom}
\emph{ReactJS, Virtual DOM and Internals}, 

\url{https://reactjs.org/docs/faq-internals.html}

 \bibitem {reconcilation}
ReactJS, Reconciliation,

\url{https://reactjs.org/docs/reconciliation.html}

 \bibitem {gnu-makefile}
GNU.org, 
\emph{Makefile},

\url{https://www.gnu.org/software/make/manual/html_node/Introduction.html}

 \bibitem {npm}
 NpmJs.com,
\emph{npm | build amazing things},

\url{https://www.npmjs.com/}

 \bibitem {react-perf}
 ReactJS,
\emph{Optimizing Performance},

\url{https://reactjs.org/docs/optimizing-performance.html}

 \bibitem {kontenery}
magnifier.pl,
\emph{Konteneryzacja - czym jest i dlaczego staje się tak popularna?},
24 Październik. 2019,

\url{https://magnifier.pl/konteneryzacja-docker-kubernetes/}

 \bibitem {promise}
MDN web docs,
\emph{Promise - JavaScript},

\url{https://developer.mozilla.org/en/docs/Web/JavaScript/Reference/Global_Objects/Promise}

 \bibitem {web-server}
MDN web docs,
\emph{What is a web server?},
18 Stycznia 2019,

\url{https://developer.mozilla.org/en-US/docs/Learn/Common_questions/What_is_a_web_server}

 \bibitem {docker}
Docker Docs,
\emph{Docker},

\url{https://docs.docker.com/get-started/#docker-concepts}

 \bibitem {nginx-windows}
Nginx,
\emph{Nginx for Windows},

\url{http://nginx.org/en/docs/windows.html}

 \bibitem {nginx-linux}
Nginx, 
\emph{Linux packages},

\url{http://nginx.org/en/linux_packages.html}

 \bibitem {docker-alpine}
Wiki Alpine Linux, 
\emph{Docker},

\url{https://wiki.alpinelinux.org/wiki/Docker}

\bibitem {ubuntu}
Docker Hub,
\emph{Ubuntu},

\url{https://hub.docker.com/_/ubuntu}


\bibitem{docker-layers}
Neo Kobo,
\emph{Docker Layered Environment},
12 Marzec 2017,

\url{http://neokobo.blogspot.com/2017/03/docker-layered-environment.html}

\bibitem{app-php-calendar-pro}
\url{https://www.apphp.com/php-calendar/index.php?page=examples}

\bibitem {rail-model}
Meggin Kearney, Addy Osmani, Kayce Basques, Jason Miller,
\emph{Measure Performance with the RAIL Model | Web Fundamentals},
12 Luty. 2019,

\url{https://developers.google.com/web/fundamentals/performance/rail}

 \bibitem {ruby}
Jesus Castello, 
\emph{Ruby Templating Engines: ERB, HAML \& Slim - RubyGuides},

\url{https://www.rubyguides.com/2018/11/ruby-erb-haml-slim/}

 \bibitem {threads}
Whatsabyte,
\emph{What Are Threads in a Processor?},
24 Sierpień 2018,

\url{https://whatsabyte.com/blog/processor-threads/}

\bibitem{php-dynamic-app}
\url{https://phppot.com/demo/php-calendar-event-management-using-fullcalendar-javascript-library/}

 \bibitem {flaky-cypress}
Dimiter Petrov,
\emph{A tale of flaky Cypress tests},
24 Październik 2019,

\url{https://dimiterpetrov.com/blog/a-tale-of-flaky-cypress-tests/}

 \bibitem {selenium}
Selenium Dev,
\emph{The Selenium project and tools},

\url{https://www.selenium.dev/documentation/en/introduction/the_selenium_project_and_tools/}

 \bibitem {docker-compose}
Docker Docs,
\emph{Overview of Docker Compose},

\url{https://docs.docker.com/compose/.}

 \bibitem {docker-compose-network}
Docker Documentation,
\emph{Networking in Compose}.

\url{https://docs.docker.com/compose/networking/}

 \bibitem {mozilla-memory}
MDN - Mozilla,
\emph{Memory Management},
4 Marca 2020,

\url{https://developer.mozilla.org/en-US/docs/Web/JavaScript/Memory_Management}

 \bibitem {mozilla-perf}
MDN - Mozilla,
\emph{Performance - Web APIs},
30 Styczeń 2020,

\url{https://developer.mozilla.org/en-US/docs/Web/API/Performance}

\bibitem {mozilla-high-res-api}
MDN - Mozilla,
\emph{DOMHighResTimeStamp - Web APIs},
19 Lusty 2020,

\url{https://developer.mozilla.org/en-US/docs/Web/API/DOMHighResTimeStamp}

 \bibitem {mdn-first-paint}
MDN - Mozilla,
\emph{First paint - MDN Web Docs Glossary},
10 Marca. 2020,

\url{https://developer.mozilla.org/en-US/docs/Glossary/First_paint}


\end{thebibliography}


% Opis bibliograficzny wydawnictwa zwartego (książki) składa się z następujących pozycji [7]:
% autorzy (nazwisko + inicjały imion), tytuł (kursywa bez cudzysłowu), nazwa wydawnictwa,
% miejsce wydania, rok wydania (w nawiasach). Poszczególne części opisu powinny być
% oddzielone przecinkami. Przy dużej liczbie autorów można podać dane pierwszego autora z frazą
% „et al.” [5].
% 
% Opis artykułu w czasopiśmie [2,9]: autorzy (nazwisko + inicjały imion), tytuł (kursywa bez
% cudzysłowu), nazwa czasopisma, wolumin, numer, rok wydania (w nawiasach), strony „od–do”
% przedzielone znakiem półpauzy (Alt+0150), bez spacji w środku.
% 
% Opis referatu w materiałach konferencyjnych lub rozdziału pracy zbiorowej [4]: autorzy
% referatu (nazwisko + inicjały imion), tytuł referatu (kursywa bez cudzysłowu), słowo (w:),
% redaktorzy pracy zbiorowej (nazwisko + inicjały imion), słowo (red.), tytuł pracy zbiorowej lub
% dane konferencji (czcionka prosta), wydawnictwo, miejsce wydania, rok wydania (w nawiasach),
% strony „od–do” przedzielone znakiem półpauzy (Alt+0150), bez spacji w środku. Jeżeli
% opisywana praca jest częścią serii wydawniczej, należy dodać jej nazwę oraz numer woluminu
% pomiędzy tytułem pracy zbiorowej i nazwą wydawnictwa [6]. Możliwe jest także zastosowanie
% skróconego opisu referatu w materiałach konferencyjnych – bez podawania redaktorów i tytułu
% pracy zbiorowej [1].
% 
% W opisie materiałów publikowanych elektronicznie (np. specyfikacji, dokumentacji
% technicznej) należy umieścić dane autora lub nazwę producenta (jeśli nie ma podanego autora),
% tytuł dokumentu, ewentualnie opis rodzaju dokumentu (np. podręcznik użytkownika), wydawcę
% i rok wydania (w nawiasach) [8]. Opis strony internetowej składa się z danych autora albo nazwy
% organizacji publikującej, tytułu dokumentu albo nazwy serwisu, jego adresu URL (bez
% podkreślenia i niebieskiego koloru), roku opublikowania oraz daty odczytania dokumentu4 [3].
% 
% 1. Agrawal R., Srikant R., Fast Algorithms for Mining Association Rules, Proceedings
% of the Twentieth International Conference on Very Large Databases, Santiago, Chile
% (1994)
% 2. Bacchus F., Grove A., Halpern J., Koller D., From statistical knowledge bases
% to degrees of belief, Artificial Intelligence, Vol. 87, No. 1–2 (1996) 75–143
% 3. Brown D., A Beginners Guide to UML. Part 1–2., Dunstan Thomas Consulting,
% http://consulting.dthomas.co.uk (2002), (odczytano 10 września 2008 r.)
% 4. Deogun J., Jiang L., Comparative Evaluation on Concept Approximation Approaches,
% (w:) Kwaśnicka H., Paprzycki M. (red.), Proceedings of the Fifth International
% Conference on Intelligent Systems Design and Applications, IEEE Computer Society
% Press, Washington, Brussels, Tokyo (2005) 438–443
% 5. Fagin R. et al., Reasoning About Knowledge, MIT Press, Cambridge, USA (1995)
% 6. Kazakov D., Kudenko D., Machine Learning and Inductive Logic Programming for
% Multi-Agent Systems, (w:) Luck M., Marik V., Stepankova O. (red.), Multi-Agent
% Systems and Applications, Lecture Notes in Artificial Intelligence (LNAI), Vol. 2086,
% Springer-Verlag, Berlin Heidelberg (2001) 246–270
% 7. Kerninghan B.W., Ritchie D.M., Język ANSI–C, Wydawnictwa Naukowo-Techniczne,
% Warszawa (1994)
% 8. Microsoft, Books On-Line, dokumentacja elektroniczna systemu MS SQL Server 2000
% Enterprise Edition, Microsoft Corporation (2000)
% 9. Perry P., Walnum C., Pierwsze kroki Telefonii GSM, Kwartalnik Elektroniki
% i Telekomunikacji, Vol. 43, No. 3 (1997) 421–430
