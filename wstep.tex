\chapter{Wstęp}

\section{Wprowadzenie}

Technologia front-end ewoluuje w zastraszającym tempie. 20 października 2010 rozpoczęła się nowa era technologii webowych wraz z wydaniem biblioteki AngularJS. Był to swoisty początek aplikacji SPA ( Single Page Application ) i całkowitej zmiany podejścia do tworzenia webowych aplikacji klienckich. 

Dzisiaj większość narzędzi jakie używamy w życiu codziennym albo w pełni znajduje się w przeglądarce albo ma już na niej swoje odpowiedniki. Znaczna część aplikacji z których korzystamy na co dzień wykorzystuje te narzędzia jako rdzeń swojego działania. Ze znanych aplikacji z jakich korzysta na co dzień użytkownik możemy wymienić jako przykład odpowiednio:
React: Facebook, Instagram, Netflix, WhatsApp
Angular 2: GitHub Community Forum, Microsoft Office Home, Google Marketing Platform
VueJS: FontAwesome, ChatWoot, Moonitor, Leave Dates


\section{Cel pracy}

W pracy tej, przeanalizowano trzy najpopularniejsze rozwiązania do tworzenia aplikacji SPA. Każde z nich pozwala na stworzenie takiej samej aplikacji z punktu widzenia końcowego użytkownika. Każda z nich jednak w inny sposób rozwiązuje problemy zarządzania zasobami oraz optymalizacji renderowania. W tym celu skonstruuje narzędzie pozwalające na badanie dynamicznych aplikacji webowych i postaram się odpowiedzieć na pytanie, czy różnica wydajności pomiędzy frameworkiami jest tak duża, że powinna mieć wpływ na wybór konkretnego rozwiązania w ogólnym przypadku?

\section{Motywacja}

Za motywacją do stworzenia tejże pracy stały dwa główne powody. Pierwszym z nich jest chęć porównania trzech najczęściej wybieranych narzędzi w środowisku front-end. Każde z nich rozwiązuje ten sam problem, jednak nie jest do jasne, jak wygląda ich wydajność w przypadku użytkowania aplikacji. Jest to bowiem problem który trudno zbadać, co prowadzi nas do drugiego powodu. Badanie aplikacji typu SPA w momencie ich działania jest niezwykle trudnym zadaniem, dlatego też, celem praktycznym pracy jest stworzenie narzędzia do rozwiązania tego problemu.


\section{Zakres}

Zadanie będzie polegać na zaprojektowaniu odpowiedniej aplikacji która zawiera kilka znanych problemów wydajnościowych występujących w świecie aplikacji SPA. Następnie implementacja takiej aplikacji w każdym z wybranych rozwiązań. Kolejno na specjalnie stworzonej maszynie wirtualnej z ograniczonymi zasobami, wykonane zostaną testy wydajnościowe za pomocą wbudowanego profilera. 

Mając już konkretne dane, wyniki zostaną przeanalizowane porównując które rozwiązanie jest najwydajniejsze w konkretnym przypadku co pozwoli na odpowiedź na pytanie: czy różnica wydajności pomiędzy narzędziami jest tak duża, że powinna mieć wpływ na wybór konkretnego rozwiązania w ogólnym przypadku?
