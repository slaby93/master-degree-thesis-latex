\chapter{Podsumowanie}

Celem pracy było stworzenie narzędzia pozwalającego na pomiar akcji aplikacji zachodzących już po załadowaniu strony oraz pomiar różnych interakcji ze stroną na przykładzie trzech wiodących rozwiązań na rynku.
Stworzenie narzędzia było niemałym wyzwaniem, gdyż problem pomiaru dynamicznych aplikacji nie jest problemem trywialnym i wymaga szeregu narzędzi współpracujących ze sobą.
Powoduje to także, iż sam pomiar będzie zanieczyszczony przez niedokładności narzędzie Selenium.
Jesteśmy w stanie jednak przeciwdziałać niektórym czynnikom mającym negatywny wpływ na pomiar, dzięki zastosowaniu warstwy izolacja systemu badającego od faktycznego systemu operacyjnego hosta badania dzięki użyciu platformy Docker.
Także poprawne przygotowanie badania na przykład poprzez przeładowanie strony i wyczyszczenie pamięci podręcznej przeglądarki poprawia stabilność i dokładność testów.
Istotnym podczas testów jest wielokrotny pomiar wartości tak, abyśmy mogli wyliczyć graniczne wartości pomiarów. Dzięki temu i wiedzy uzyskanej podczas studiów możemy przeprowadzić poprawnę analizę wyników.

Same badania wykazały, iż Angular2 najczęściej jest najszybszym narzędziem, jednak bardzo często odchylenie standardowe wskazywało, iż także istnieje duży rozrzut wartości wokół średniej.
React jest najbardziej stabilnym rozwiązaniem, gdyż w dwóch badaniach uzyskał odchylenie standardowe na poziomie poniżej 2 milisekund. W reszcie przypadków, różnica pomiędzy Vue i Reactem mieściła się w granicy błędu pomiarowego.

Podsumowując -najważniejszym wnioskiem który udało się potwierdzić dzięki przeprowadzonym badaniom jest wykazanie, iż różnica pomiędzy przedstawionymi rozwiązaniami nie przekracza 20 milisekund dla pojedynczego zadania.
Jest to istotny argument podczas dyskusji nad przewagą konkretnych narzędzi pomiędzy sobą.
Jest to wniosek zgodny także z pracami cytowanych autorów, iż najistotniejsze jest środowisko programistyczne dostarczone wraz z narzędziem oraz popularność narzędzia która bezpośrednio ma wpływ na przykład
na ilość informacji dostępnych w internecie na temat częstych problemów niżeli wydajność pojedynczego narzędzia w skali mikro.


%czyści puste strony
\let\cleardoublepage\clearpage